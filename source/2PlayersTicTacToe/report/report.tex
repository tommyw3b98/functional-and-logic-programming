\documentclass{article}
\usepackage[italian]{babel}
\usepackage[utf8]{inputenc}
\usepackage[T1]{fontenc}
\usepackage{verbatim}
\usepackage{listingsutf8}
\usepackage{color}

\definecolor{dkgreen}{rgb}{0,0.6,0}
\definecolor{gray}{rgb}{0.5,0.5,0.5}
\definecolor{mauve}{rgb}{0.58,0,0.82}

\lstset
{
    language=Haskell,
    language=Prolog,
    aboveskip=3mm,
    belowskip=3mm,
    showstringspaces=false,
    columns=flexible,
    basicstyle={\small\ttfamily},
    numbers=none,
    numberstyle=\tiny\color{gray},
    keywordstyle=\color{blue},
    commentstyle=\color{dkgreen},
    stringstyle=\color{mauve},
    breaklines=true,
    breakatwhitespace=true,
    tabsize=3,
    inputencoding=utf8,
}

%pagina iniziale
\title{Progetto~di~programmazione~funzionale~e~logica}
\author{Tommaso Cicco}
\date{\small Anno accademico 2021/2022}

\begin{document}
\maketitle
\newpage
\tableofcontents
\newpage
 
%specifica   
\section{Specifica del Problema}

Il gioco del tris si svolge su una tabella 3 × 3 in cui si alternano le mosse di due giocatori indicati rispettivamente con X e O. 
Il gioco si conclude con la vittoria di X (risp. O) se sono stati messi tre simboli X (risp. O) sulla stessa riga, colonna o diagonale. 
Scrivere un programma che acquisisce da tastiera una mossa alla volta mostrando sullo schermo il contenuto aggiornato della tabella 
e determina la vittoria di uno dei due giocatori o il pareggio stampando sullo schermo un apposito messaggio.
\newpage
\section{Analisi del problema}

\subsection{Analisi dei dati di input}

Ad ogni turno di gioco, i dati in ingresso sono rappresentati dalla scelta di una casella sulla tabella da parte di uno dei giocatori.
\subsection{Analisi dei dati di output}

Ad ogni turno di gioco, i dati in uscita sono rappresentati dallo stato della tabella, e, nel caso in cui si raggiunga una delle condizioni di fine partita, dal messaggio che comunica ai giocatori il vincitore o il pareggio.
\subsection{Relazioni intercorrenti tra i dati}

\begin{itemize}
    \item Lo stato della tabella è aggiornato ad ogni turno a seconda della scelta del giocatore. Il nuovo stato sarà uguale a quello precedente con il simbolo del giocatore presente nella casella scelta.
    \item La fine di una partita è determinata dal raggiungimento di una condizione di vittoria oppure di pareggio.
    \item La condizione di vittoria è raggiunta nel momento in cui sulla tabella sono presenti tre simboli identici sulla stessa riga, colonna o diagonale.
    \item La condizione di pareggio è raggiunta nel momento in cui non sono più presenti caselle libere e non è stata raggiunta una condizione di vittoria.
\end{itemize}
\newpage

%progettazione
\section{Progettazione dell'algoritmo}

\subsection{Scelte di progetto}

\begin{itemize}
    \item Per rappresentare la tabella di gioco si utilizzerà il tipo di dato lista, i cui elementi rappresenteranno le nove caselle ordinate a partire dalla prima riga procedendo da sinistra verso destra.
    \item La lista sarà inizialmente costituita da elementi "segnaposto", i quali rappresentano una casella libera; alla fine di ogni turno sarà generata una nuova lista con il "segnaposto" corrispondente alla casella scelta sostituito dal simbolo del giocatore cha ha appena effettuato una mossa.
    \item Per consentire la scelta di una casella, le righe della tabella saranno indicizzate da una lettera (A, B, C) e le colonne da un numero (1, 2, 3); una mossa valida sarà rappresentata dagli indici di riga e di colonna di una casella libera (esempio: A2).
    \item Al termine di ogni partita, sarà data la possibilità ai giocatori di iniziarne una nuova senza dover riavviare il programma.
\end{itemize}
\subsection{Passi dell'algoritmo}

\begin{enumerate}
    \item Acquisire la mossa di un giocatore:
    \begin{itemize}
        \item Controllare che la mossa inserita rappresenti una casella (A1 - C3).
        \item Controllare che la casella scelta dal giocatore sia libera.
        \item Quando un controllo fallisce, l'acquisizione è ripetuta.
    \end{itemize}
    \item Aggiornare lo stato della tabella:
    \begin{itemize}
        \item Generare una nuova lista sostituendo il "segnaposto" corrispondente alla casella scelta con il simbolo del giocatore corrente.
    \end{itemize}
    \item Controllare il raggiungimento di una condizione di vittoria:
     \begin{itemize}
        \item Controllare ogni trio di elementi della lista che costituisce una riga, colonna, o diagonale sulla tabella.
        \item Se è presente un trio di elementi identici (diversi dal "segnaposto"), il giocatore ha vinto.
    \end{itemize}
    \item Se non è stata raggiunta una condizione di vittoria, controllare il raggiungimento di una condizione di pareggio:
    \begin{itemize}
        \item Controllare che la lista non contenga elementi "segnaposto".
    \end{itemize}
    \item Visualizzare la tabella aggiornata.
    \item Se non è stata raggiunta una condizione di fine partita, acquisire la mossa del giocatore successivo (punto 1).
    \item Se è stata raggiunta una condizione di fine partita, visualizzare il messaggio adatto.
    \item Chiedere all'utente se vuole giocare di nuovo:
    \begin{itemize}
        \item Se la risposta è affermativa, rigenerare la lista iniziale e continuare l'esecuzione dal primo punto.
        \item In caso contrario, terminare l'esecuzione.
    \end{itemize}
\end{enumerate}
\newpage

%implementazione
\section{Implementazione dell'algoritmo}

\subsection{Implementazione in Haskell}

\lstinputlisting[language=Haskell]{../haskell/2PlayersTicTacToe.hs}
\newpage
\subsection{Implementazione in Prolog}

\lstinputlisting[language=Prolog]{../prolog/2PlayersTicTacToe.pl}
\newpage

%testing haskell
\section{Testing dell'implementazione in Haskell}

\subsection{Acquisizione corretta (mossa del giocatore X)}
    \begin{verbatim}
    Player X enter a move (A1 - C3): 
    b2
         1     2     3

    A    -  |  -  |  -  
       _____|_____|_____
            |     |     
    B    -  |  X  |  -  
       _____|_____|_____
            |     |     
    C    -  |  -  |  -  

    Player O enter a move (A1 - C3): 
    \end{verbatim}

\subsection{Acquisizione corretta (mossa del giocatore O)}
    \begin{verbatim}
    Player O enter a move (A1 - C3): 
    c1
         1     2     3

    A    -  |  -  |  -  
       _____|_____|_____
            |     |     
    B    -  |  X  |  -  
       _____|_____|_____
            |     |     
    C    O  |  -  |  -  

    Player X enter a move (A1 - C3): 
    \end{verbatim}

\subsection{Validazione dell'input (mossa inesistente)}
    \begin{verbatim}
    Player X enter a move (A1 - C3): 
    A-1

    Invalid move, try again!

    Player X enter a move (A1 - C3): 
    2,345

    Invalid move, try again!

    Player X enter a move (A1 - C3): 
    \end{verbatim}

\subsection{Validazione dell'input (casella occupata)}
    \begin{verbatim}
    Player X enter a move (A1 - C3): 
    a1
         1     2     3

    A    X  |  -  |  -  
       _____|_____|_____
            |     |     
    B    -  |  X  |  -  
       _____|_____|_____
            |     |     
    C    O  |  -  |  -  

    Player O enter a move (A1 - C3): 
    a1

    Invalid move, try again!

    Player O enter a move (A1 - C3): 
    \end{verbatim}

\subsection{Vittoria del giocatore X}
    \begin{verbatim}
    Player X enter a move (A1 - C3): 
    a3
         1     2     3

    A    X  |  X  |  X  
       _____|_____|_____
            |     |     
    B    O  |  X  |  O  
       _____|_____|_____
            |     |     
    C    O  |  -  |  -  

    Player X won!

    Play again? (y/n):
    \end{verbatim}

\subsection{Vittoria del giocatore O}
    \begin{verbatim}
    Player O enter a move (A1 - C3): 
    c1
         1     2     3

    A    X  |  -  |  O  
       _____|_____|_____
            |     |     
    B    X  |  O  |  -  
       _____|_____|_____
            |     |     
    C    O  |  X  |  -  

    Player O won!

    Play again? (y/n):
    \end{verbatim}

\subsection{Pareggio}
    \begin{verbatim}
    Player X enter a move (A1 - C3): 
    c3
         1     2     3

    A    X  |  O  |  X  
       _____|_____|_____
            |     |     
    B    X  |  O  |  O  
       _____|_____|_____
            |     |     
    C    O  |  X  |  X  

    It's a tie!

    Play again? (y/n):
    \end{verbatim}

\subsection{Vittoria con l'ultima casella}
    \begin{verbatim}
    Player X enter a move (A1 - C3): 
    c3
         1     2     3

    A    X  |  O  |  X  
       _____|_____|_____
            |     |     
    B    O  |  O  |  X  
       _____|_____|_____
            |     |     
    C    O  |  X  |  X  

    Player X won!

    Play again? (y/n):
    \end{verbatim}

\subsection{Inizio di nuova partita}
    \begin{verbatim}
    Player X won!

    Play again? (y/n):
    y
    Welcome to tic tac toe, good luck and have fun!

         1     2     3

    A    -  |  -  |  -  
       _____|_____|_____
            |     |     
    B    -  |  -  |  -  
       _____|_____|_____
            |     |     
    C    -  |  -  |  -  

    Player X enter a move (A1 - C3): 
    \end{verbatim}

\subsection{Uscita dal programma}
    \begin{verbatim}
    Player X enter a move (A1 - C3): 
    c1
         1     2     3

    A    X  |  O  |  X  
       _____|_____|_____
            |     |     
    B    O  |  X  |  O  
       _____|_____|_____
            |     |     
    C    X  |  -  |  -  

    Player X won!

    Play again? (y/n):
    n
    Bye Bye!
    *Main> 
    \end{verbatim}
\newpage

%testing prolog
\section{Testing dell'implementazione in Prolog}

\subsection{Acquisizione corretta (mossa del giocatore X)}
    \begin{verbatim}
    Player X enter a move (a1 - c3): b2.
         1     2     3

    A    -  |  -  |  -  
       _____|_____|_____
            |     |     
    B    -  |  X  |  -  
       _____|_____|_____
            |     |     
    C    -  |  -  |  -  

    Player O enter a move (a1 - c3): 
    \end{verbatim}

\subsection{Acquisizione corretta (mossa del giocatore O)}
    \begin{verbatim}
    Player O enter a move (a1 - c3): a1.
         1     2     3

    A    O  |  -  |  -  
       _____|_____|_____
            |     |     
    B    -  |  X  |  -  
       _____|_____|_____
            |     |     
    C    -  |  -  |  -  

    Player X enter a move (a1 - c3): 
    \end{verbatim}

\subsection{Validazione dell'input (mossa inesistente)}
    \begin{verbatim}
    Player X enter a move (a1 - c3): a32.

    Invalid move, try again!

    Player X enter a move (a1 - c3): 23451.  

    Invalid move, try again!

    Player X enter a move (a1 - c3): ciao.

    Invalid move, try again!

    Player X enter a move (a1 - c3): 
    \end{verbatim}

\subsection{Validazione dell'input (casella occupata)}
    \begin{verbatim}
    Player X enter a move (a1 - c3): a2
    .
         1     2     3

    A    O  |  X  |  -  
       _____|_____|_____
            |     |     
    B    -  |  X  |  -  
       _____|_____|_____
            |     |     
    C    -  |  -  |  -  

    Player O enter a move (a1 - c3): a2.

    Invalid move, try again!

    Player O enter a move (a1 - c3): 
    \end{verbatim}

\subsection{Vittoria del giocatore X}
    \begin{verbatim}
    Player X enter a move (a1 - c3): c1.
         1     2     3

    A    X  |  O  |  X  
       _____|_____|_____
            |     |     
    B    O  |  X  |  -  
       _____|_____|_____
            |     |     
    C    X  |  -  |  O  

    Player X won!
    Play again? (y/n): 
    \end{verbatim}

\subsection{Vittoria del giocatore O}
    \begin{verbatim}
    Player O enter a move (a1 - c3): c1.
         1     2     3

    A    O  |  X  |  X  
       _____|_____|_____
            |     |     
    B    O  |  X  |  -  
       _____|_____|_____
            |     |     
    C    O  |  -  |  -  

    Player O won!
    Play again? (y/n): 
    \end{verbatim}

\subsection{Pareggio}
    \begin{verbatim}
    Player X enter a move (a1 - c3): c3.
         1     2     3

    A    X  |  X  |  O  
       _____|_____|_____
            |     |     
    B    O  |  O  |  X  
       _____|_____|_____
            |     |     
    C    X  |  O  |  X  

    It's a tie!
    Play again? (y/n): 
    \end{verbatim}

\subsection{Vittoria con l'ultima casella}
    \begin{verbatim}
    Player X enter a move (a1 - c3): c3.
         1     2     3

    A    X  |  O  |  X  
       _____|_____|_____
            |     |     
    B    O  |  X  |  O  
       _____|_____|_____
            |     |     
    C    O  |  X  |  X  

    Player X won!
    Play again? (y/n): 
    \end{verbatim}

\subsection{Inizio di nuova partita}
    \begin{verbatim}
    Player X won!
    Play again? (y/n): y.

    Welcome to tic tac toe, good luck and have fun!
    Warning: input must be lowercase and end with a period

         1     2     3

    A    -  |  -  |  -  
       _____|_____|_____
            |     |     
    B    -  |  -  |  -  
       _____|_____|_____
            |     |     
    C    -  |  -  |  -  

    Player X enter a move (a1 - c3):
    \end{verbatim}

\subsection{Uscita dal programma}
    \begin{verbatim}
    Player X enter a move (a1 - c3): c1.

         1     2     3

    A    X  |  O  |  X  
       _____|_____|_____
            |     |     
    B    O  |  X  |  O  
       _____|_____|_____
            |     |     
    C    X  |  -  |  -  

    Player X won!
    Play again? (y/n): n.

    Bye Bye!

    true ? 

    (5 ms) yes
    | ?- 
    \end{verbatim}
\end{document}
